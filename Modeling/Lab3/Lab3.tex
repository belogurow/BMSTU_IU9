%% It is just an empty TeX file.
%% Write your code here.
% !TEX encoding = UTF-8 Unicode
\documentclass[a4paper, 12pt]{article}   	% use "amsart" instead of "article" for AMSLaTeX format
\usepackage[left=20mm, top=15mm, right=10mm, bottom=15mm]{geometry}    

            
\usepackage[parfill]{parskip}    		% Activate to begin paragraphs with an empty line rather than an indent
\usepackage{graphicx}				% Use pdf, png, jpg, or eps§ with pdflatex; use eps in DVI mode
\usepackage[14pt]{extsizes}
\usepackage{setspace,amsmath}
\usepackage{ dsfont }
\usepackage{graphicx}
\renewcommand{\labelenumii}{\theenumii}
\renewcommand{\theenumii}{\theenumi.\arabic{enumii}.}
\usepackage{amsmath,amssymb}
\usepackage[unicode]{hyperref}

\usepackage{xcolor}
\usepackage{color}
\usepackage{minted}
\usepackage{caption}

\usepackage{array}
\newcolumntype{P}[1]{>{\centering\arraybackslash}p{#1}}

\usepackage{cmap} % Улучшенный поиск русских слов в полученном pdf-файле
\usepackage[T2A]{fontenc} % Поддержка русских букв
\usepackage[utf8]{inputenc} % Кодировка utf8
\usepackage[english, russian]{babel} % Языки: русский, английский

								% TeX will automatically convert eps --> pdf in pdflatex		
\usepackage{amssymb}

\begin{document}
\begin{titlepage}

\thispagestyle{empty}

\begin{center}
Федеральное государственное бюджетное образовательное учреждение высшего профессионального образования Московский государственный технический университет имени Н.Э. Баумана
\end{center}


\vfill

\centerline{\large{Лабораторная работа №3}}

\centerline{\large{«Имитационное моделирование системы массового обслуживания}} 
\centerline{\large{в среде GPSS»}}

\centerline{\large{по курсу}}
\centerline{\large{«Моделирование»}}


\vfill

Студент группы ИУ9-82 \hfill Белогуров А.А.

Преподаватель \hfill Домрачева А.Б.
\vfill

\centerline{Москва, 2018}
\clearpage
\end{titlepage}

\newpage
\setcounter{page}{2}

\tableofcontents

\newpage

\section{Постановка задачи}
    Изучение основ имитационного моделирования в среде GPSS World на примере простейших одноканальных систем массового обслуживания (СМО) с неограниченной очередью.
    
    Смоделировать работу простейшей СМО на примере поступления заявок на бронирование номеров в отеле без ограничения длины очереди для $N = 700$ зявок. Поток поступления заявок на электронную почту распределен по закону Пуассона с интенсивностью $L = 2.15$ заявки на бронирование каждые 10 минут, а время обработки заявки администратором отеля распределено экспоненциально со средней интенсивностью обработки $M=2.45$ заявки каждые 10 минут.
    
    Описать модель с помощью терминов систем массового обслуживания.
    
    Определить следующие характеристики системы:
    \begin{enumerate}
        \item коэффициент загрузки устройства (в \text{\%});
        \item среднее число находящихся в очереди заявок;
        \item среднее продолжительность пребывания заявки в очереди.
    \end{enumerate}
    
    Сравнить их со значениями, определенными аналитически по формулам:
    \begin{enumerate}
        \item $r = \frac{L}{M}$
        \item $avgNumber = \frac{r^2}{1 - r}$
        \item $avgQueue = \frac{r^2}{L(1-r)}$
    \end{enumerate}

\newpage

\section{Теоретические сведения}
    \textit{\textbf{\underline{Определение 1.}} Система массового обслуживания (СМО) — система, которая производит обслуживание поступающих в неё требований. Обслуживание требований в СМО выполняется обслуживающими приборами. }
    
    \subsection{Основные определения СМО.}
    
        \textit{\textbf{\underline{Определение 2.}} Требование (заявка) — запрос на обслуживание. }
        
        \noindent\textit{\textbf{\underline{Определение 3.}} Входящий поток требований — совокупность требований, поступающих в СМО. }
        
        \noindent\textit{\textbf{\underline{Определение 4.}} Время обслуживания — период времени, в течение которого обслуживается требование. }
        
        \noindent\textit{\textbf{\underline{Определение 5.}} Математическая модель СМО — это совокупность математических выражений, описывающих входящий поток требований, процесс обслуживания и их взаимосвязь.. }
    
    \subsection{Определение текущей задачи в терминах СМО.}
        \begin{itemize}
            \item Заявкой является письмо на электронную почту с просьбой запронировать номер в отеле на определенный период времеини;
            \item Входящим потоком требований является ограничение на обработку 7000 заявок.
            \item Время обслуживания - среднее время обработки заявки на бронироание.
            \item Математическая модель СМО является программой, написанной на языке моделирования GPSS.
        \end{itemize}
    

\newpage

\section{Практическая реализация}

\textbf{Листинг 1.} Описание имитационной модели на языке моделирования GPSS.
\begin{minted}[frame=single, framesep=10pt, fontsize = \footnotesize, linenos=true, breaklines]{text}
OPERATOR STORAGE 1                ; количество администраторов

GENERATE .465,FN$EXPON,,7000      ; поступление заявок(2.15 заявок каждый 10 минут), всего 7000 заявок

QUEUE ocher                       ; занятие очереди на обработку заявки

ENTER OPERATOR                    ; поступление заявки на электронную почту

DEPART ocher                      ; уход заказа из очереди

ADVANCE .408,FN$EXPON             ; время обработки заявки(2.45 заявок каждые 10 минут)

LEAVE OPERATOR                    ; завершение обработки заявки на бронирование номера

TERMINATE

GENERATE 3300

TERMINATE 1

EXPON FUNCTION RN2,C24            ;описание функции EXPON
0,0/0.1,0.104/0.2,0.222/0.3,0.355/0.4,0.509
0.5,0.69/0.6,0.915/0.7,1.2/0.75,1.38/.8,1.6
.84,1.83/.88,2.12/.9,2.3/.92,2.52/.94,2.81
.95,2.99/.96,3.2/.97,3.5/.98,3.9/.99,4.6
.995,5.3/.998,6.2/.999,7/.9998,8
\end{minted}

\newpage
\section{Результаты}
    Ниже приведен результат работы программы из \textbf{Листинга 1}.
    
    \noindent\textbf{Листинг 2.} Описание модели на языке GPSS.
\begin{minted}[frame=single, framesep=6pt, fontsize = \footnotesize, linenos=true, breaklines]{text}
              GPSS World Simulation Report - Untitled Model 1.1.1


                   Tuesday, March 20, 2018 10:35:58  

           START TIME           END TIME  BLOCKS  FACILITIES  STORAGES
                0.000           3300.000     9        0          1


              NAME                       VALUE  
          EXPON                       10001.000
          OCHER                       10002.000
          OPERATOR                    10000.000


 LABEL              LOC  BLOCK TYPE     ENTRY COUNT CURRENT COUNT RETRY
                    1    GENERATE          7000             0       0
                    2    QUEUE             7000             0       0
                    3    ENTER             7000             0       0
                    4    DEPART            7000             0       0
                    5    ADVANCE           7000             0       0
                    6    LEAVE             7000             0       0
                    7    TERMINATE         7000             0       0
                    8    GENERATE             1             0       0
                    9    TERMINATE            1             0       0


QUEUE              MAX CONT. ENTRY ENTRY(0) AVE.CONT. AVE.TIME   AVE.(-0) RETRY
 OCHER              40    0   7000    884     5.414      2.552      2.921   0


STORAGE            CAP. REM. MIN. MAX.  ENTRIES AVL.  AVE.C. UTIL. RETRY DELAY
 OPERATOR            1    1   0     1     7000   1    0.871  0.871    0    0


FEC XN   PRI         BDT      ASSEM  CURRENT  NEXT  PARAMETER    VALUE
  7002    0        6600.000   7002      0      8
\end{minted}
    

    \textbf{Анализ результата моделирования}:
    \begin{enumerate}
        \item коэффициент загрузки устройства $= 87.1 \%$
        \item среднее число находящихся в очереди заявок $= 5.4$ заявки;
        \item средняя продолжительность пребывания заявки в очереди $= 2.6 * 10$ мин $= 20.6$ мин 
    \end{enumerate}
    
    \textbf{Аналитические значения}:
    \begin{enumerate}
        \item $r = \frac{L}{M} = 87.7 \%$
        \item $avgNumber = \frac{r^2}{1 - r} = 6.2$
        \item $avgQueue = \frac{r^2}{L(1-r)} = 2.8$
    \end{enumerate}


\newpage
\section{Вывод}
    В ходе лабораторной работы был изучен язык моделирования GPSS, который использовался для реализации системы массового обслуживания, а в данном случае для моделирования бронирования номеров в отеле.
    
    Стоит отметить, что с помощью данного языка возможно моделировать большое количество различных ситуаций производства, а одна программа может успешно использоваться для разных сфер деятельности.


\end{document} 














