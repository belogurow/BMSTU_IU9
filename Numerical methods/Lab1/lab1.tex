% !TEX encoding = UTF-8 Unicode
\documentclass[a4paper, 12pt]{article}   	% use "amsart" instead of "article" for AMSLaTeX format
\usepackage[left=20mm, top=15mm, right=15mm, bottom=15mm, nohead, footskip=10mm]{geometry}    

            
\usepackage[parfill]{parskip}    		% Activate to begin paragraphs with an empty line rather than an indent
\usepackage{graphicx}				% Use pdf, png, jpg, or eps§ with pdflatex; use eps in DVI mode
\usepackage[14pt]{extsizes}
\usepackage{setspace,amsmath}
\usepackage{ dsfont }
\usepackage{amsmath,amssymb}
\usepackage{hyperref}

\usepackage{xcolor}
\usepackage{color}
\usepackage{listings}
\usepackage{setspace}
\definecolor{Code}{rgb}{0,0,0}
\definecolor{Decorators}{rgb}{0.5,0.5,0.5}
\definecolor{Numbers}{rgb}{0.5,0,0}
\definecolor{MatchingBrackets}{rgb}{0.25,0.5,0.5}
\definecolor{Keywords}{rgb}{0,0,1}
\definecolor{self}{rgb}{0,0,0}
\definecolor{Strings}{rgb}{0,0.63,0}
\definecolor{Comments}{rgb}{0,0.63,1}
\definecolor{Backquotes}{rgb}{0,0,0}
\definecolor{Classname}{rgb}{0,0,0}
\definecolor{FunctionName}{rgb}{0,0,0}
\definecolor{Operators}{rgb}{0,0,0}
\definecolor{Background}{rgb}{0.98,0.98,0.98}
\lstdefinelanguage{Python}{
numbers=left,
numberstyle=\footnotesize,
numbersep=1em,
xleftmargin=0em,
xrightmargin=-3em,
framextopmargin=2em,
framexbottommargin=2em,
showspaces=false,
showtabs=false,
showstringspaces=false,
frame=l,
tabsize=2,
% Basic
basicstyle=\ttfamily\small\setstretch{1},
%backgroundcolor=\color{Background},
% Comments
commentstyle=\color{Comments}\slshape,
% Strings
stringstyle=\color{Strings},
morecomment=[s][\color{Strings}]{"""}{"""},
morecomment=[s][\color{Strings}]{'''}{'''},
% keywords
morekeywords={import,from,class,def,for,while,if,is,in,elif,else,not,and,or,print,break,continue,return,True,False,None,access,as,,del,except,exec,finally,global,import,lambda,pass,print,raise,try,assert},
keywordstyle={\color{Keywords}\bfseries},
% additional keywords
morekeywords={[2]@invariant,pylab,numpy,np,scipy},
keywordstyle={[2]\color{Decorators}\slshape},
emph={self},
emphstyle={\color{self}\slshape},
texcl=true,
%
}
\usepackage{caption}
\DeclareCaptionFont{white}{\color{white}}
\DeclareCaptionFormat{listing}{\colorbox{gray}{\parbox{\textwidth}{#1#2#3}}}
\captionsetup[lstlisting]{format=listing,labelfont=white,textfont=white} 


\usepackage{cmap} % Улучшенный поиск русских слов в полученном pdf-файле
\usepackage[T2A]{fontenc} % Поддержка русских букв
\usepackage[utf8]{inputenc} % Кодировка utf8
\usepackage[english, russian]{babel} % Языки: русский, английский


 

								% TeX will automatically convert eps --> pdf in pdflatex		
\usepackage{amssymb}



\begin{document}
\begin{titlepage}

\thispagestyle{empty}

\begin{center}
Федеральное государственное бюджетное образовательное учреждение высшего профессионального образования Московский государственный технический университет имени Н.Э. Баумана
\end{center}


\vfill

\centerline{\large{Лабораторная работа №1}}
\centerline{\large{по дисциплине}}
\centerline{\large{<<Численные методы>>}}

\vfill

Студент группы ИУ9-62 \hfill Белогуров А.А.

Преподаватель \hfill Домрачева А.Б.
\vfill

\centerline{Москва, 2017}
\clearpage
\end{titlepage}

\newpage
\setcounter{page}{2}
 \textbf{Постановка задачи:}

Дано: $A\overline{x}=\overline{d}$, где $A \in \mathds{R}^{n \times n}$ и $ \overline{x},\overline{d} \in \mathds{R}^{n}$. Найти решение СЛАУ с помощью методов: Гаусса и Зейделя. В качестве исходных данных взять следующие матрицы \eqref{eq:0}:
\begin{equation}\label{eq:0}
A\overline{x}=\overline{d}
\quad   \Leftrightarrow \quad
\left(\begin{array}{cccc} 
4 & 1 & 0 & 0 \\
1 & 4 & 1 & 0 \\
0 & 1 & 4 & 1 \\
0 & 0 & 1 & 4  
\end{array}\right)
\left(\begin{array}{c} 
x_1 \\
x_2 \\
x_3 \\
x_4 \\
\end{array}\right)
 =
 \left(\begin{array}{c} 
5 \\
6 \\
6 \\
5 \\
\end{array}\right)
\end{equation}
\\

\textbf{Теоретические сведения:}

Решение СЛАУ методом Гаусса:

\textbf{1.}  Составить расширенную матрицу $(A|d)$ и привести её к ступенчатому виду\eqref{matrixp1}:
\begin{equation}\label{matrixp1}
 (\widetilde{A}|\widetilde{d})=\left(\begin{array}{ccccc|c} 
1 & \widetilde{a_{12}} & \widetilde{a_{13}} & \ldots &\widetilde{a_{1n}} & \widetilde{d_{1}} \\ 
0 & 1 & \widetilde{a_{23}} & \ldots &\widetilde{a_{2n}} & \widetilde{d_{2}} \\ 
  \vdots  & \vdots  & \vdots & \ddots & \vdots  & \vdots\\
 0 & 0 & 0 & \ldots& 1 & \widetilde{d_{n}} 
\end{array}\right)
\end{equation}


\textbf{2.} Найти приближенный вектор $\overline{x}^*$. При вычислении $\widetilde{a_{ij}}$ и $\widetilde{d_i}$ с плавающей точкой возникает погрешность $\overline{e}$ \eqref{eq:9}:
\[
  \begin{array}{l@{\quad}cr@{}l}
    && A\overline{x}^* & {}= \overline{d}^* \\
    \text{--} && A\overline{x} & {}= \overline{d} \\ \cline{2-4}
    && A(\overline{x}^*-\overline{x}) & {}= (\overline{d}^*-\overline{d})
  \end{array}
\]
или:
\begin{equation}\label{eq:9}
 A\overline{e} = \overline{r} \\
  \quad
  \Leftrightarrow
  \quad 
   \boxed{\overline{e} = A^{-1} \overline{r}} 
\end{equation}

\textbf{3.} Тогда исходный вектор $ \overline{x} = \overline{x}^* - \overline{e}$.
\\

Решение СЛАУ методом Зейделя:

\textbf{1.} Преобразовать систему $A\overline{x}=\overline{d}$ к виду $x = B\overline{x} + \overline{c}$, где
\begin{equation*}
b_{ij}=-\frac{a_{ij}}{a_{ii}}, b_{ii} = 0, i \neq j
\end{equation*}
\begin{equation*}
c_i = \frac{d_i}{a_{ii}}, a_{ii} \neq 0
\end{equation*}

\textbf{2.} Задать начальное приближение решения $x^{(0)}$ произвольно, определить $k$ и малое положительное число $\varepsilon$:
\begin{equation*}
x^{(0)} = 0, k = 0, \varepsilon = 0.01
\end{equation*}

\textbf{3.} Произвести расчеты по формуле \eqref{systemp3} и найти $x^{(k+1)}$.
\begin{equation}\label{systemp3}
\begin{cases}
x_{1}^{(k+1)} = b_{12} x_{2}^{(k)} + ... + b_{1n} x_{n}^{(k)} + c_1
\\
x_{2}^{(k+1)} = b_{21} x_{1}^{(k+1)} + ... + b_{2n} x_{n}^{(k)} + c_2
\\
\vdots
\\
x_{n}^{(k+1)} = b_{n1}  x_{1}^{(k+1)} + ... + b_{nn-1} x_{n-1}^{(k+1)} + c_1
\end{cases}
\end{equation}

\textbf{4.} Если выполнено условие окончания $\Vert x^{(k-1)} - x^{(k)} \Vert \leq \varepsilon$ \eqref{normp4}, то завершить процесс и в качестве приближенного решения принять $x_{*} \cong x^{(k)}$. Иначе положить $k = k + 1$ и перейти к пункту 3. 
\begin{equation}\label{normp4}
\Vert x^{(k-1)} - x^{(k)} \Vert = \sqrt{\sum_{i=1}^{n} (x_{i}^{(k-1)} - x_{i}^{(k)})^2} \leq \varepsilon
\end{equation}
\\



\textbf{Практическая реализация:}
\hypertarget{lst:gauss}{}
\lstinputlisting[label = lst:gauss,  caption=Решение СЛАУ методом Гаусса, language=Python]{lab1.py}
\hypertarget{lst:zeidel}{}
\lstinputlisting[caption=Решение СЛАУ методом Зейделя, language=python]{lab1_2.py}


\textbf{Результаты:}

Решение СЛАУ методом Гаусса.

В качестве исходных данных возьмем матрицы \eqref{eq:0}. Тогда в результате работы программы  \hyperlink{lst:gauss}{(Листинг 1)} получены следующие значения:
\begin{equation}\label{eq:5}
e = 
\left(\begin{array}{c} 
0 \\
0 \\
0 \\
0 \\
\end{array}\right)
\quad
x =
 \left(\begin{array}{c} 
1 \\
1 \\
1 \\
1 \\
\end{array}\right)
\end{equation}
Как видно выше \eqref{eq:5}, не всегда вектор $\overline{e}$ может содержать погрешность. Для того, чтобы вектор $\overline{e}$ был ненулевым, возьмем другие исходные значения\eqref{eq:6} и получим следующие результаты\eqref{eq:7}:
\begin{equation}\label{eq:6}
A\overline{x}=\overline{d}
\quad   \Leftrightarrow \quad
\left(\begin{array}{cccc} 
1 & 2 & 0 & 0 \\
2 & -1 & -1 & 0 \\
0 & 1 & -1 & 1 \\
0 & 0 & 1 & 1  
\end{array}\right)
\left(\begin{array}{c} 
x_1 \\
x_2 \\
x_3 \\
x_4 \\
\end{array}\right)
 =
 \left(\begin{array}{c} 
5 \\
3 \\
3 \\
7 \\
\end{array}\right)
\end{equation}

\begin{equation}\label{eq:7}
e = 
\left(\begin{array}{c} 
3.947459643111668e^{-16} \\
-1.973729821555834e^{-16} \\
-9.86864910777917e^{-16} \\
9.86864910777917e^{-16} \\
\end{array}\right)
\quad
x =
 \left(\begin{array}{c} 
1 \\
2 \\
3 \\
4 \\
\end{array}\right)
\end{equation}
\\

Решение СЛАУ методом Зейделя.

Для тестирования работы программы \hyperlink{lst:zeidel}{(Листинг 2)} была взята система \eqref{eq:0}. 

Для $\varepsilon = 0.01$ было выполнено 5 итераций. Результаты практически аналогичны методу Гаусса  \eqref{eq:8}:
\begin{equation}\label{eq:8}
x =
 \left(\begin{array}{c} 
1.0003395080566406 \\
0.9997768402099609 \\
1.0000903606414795 \\
0.9999774098396301 \\
\end{array}\right)
\end{equation}
\\


\textbf{Выводы:}

В ходе выполнения лабораторной работы были рассмотрены итерационные методы решения СЛАУ: метод Гаусса и метод Зейделя, так же для каждого из них была написана реализация на языке программирования Python.

Для метода Гаусса можно отметить довольно высокую вычислительную сложность и высокое накопление погрешности в результате приведения матрицы к треугольному виду и преобразование матрицы $A$ к обратной $A^{-1}$.

Для метода Зейделя характерно довольное малое количество итераций для нахождения решения и простота вычислений. Но самое главное то, что данный метод уступает по точности методу Гаууса.



\end{document}  