% !TEX encoding = UTF-8 Unicode
\documentclass[a4paper, 12pt]{article}   	% use "amsart" instead of "article" for AMSLaTeX format
\usepackage[left=20mm, top=15mm, right=15mm, bottom=15mm, nohead, footskip=10mm]{geometry}    

            
\usepackage[parfill]{parskip}    		% Activate to begin paragraphs with an empty line rather than an indent
\usepackage{graphicx}				% Use pdf, png, jpg, or eps§ with pdflatex; use eps in DVI mode
\usepackage[14pt]{extsizes}
\usepackage{setspace,amsmath}
\usepackage{ dsfont }
\usepackage{amsmath,amssymb}
\usepackage{hyperref}

\usepackage{xcolor}
\usepackage{color}
\usepackage{listings}
\usepackage{setspace}
\definecolor{Code}{rgb}{0,0,0}
\definecolor{Decorators}{rgb}{0.5,0.5,0.5}
\definecolor{Numbers}{rgb}{0.5,0,0}
\definecolor{MatchingBrackets}{rgb}{0.25,0.5,0.5}
\definecolor{Keywords}{rgb}{0,0,1}
\definecolor{self}{rgb}{0,0,0}
\definecolor{Strings}{rgb}{0,0.63,0}
\definecolor{Comments}{rgb}{0,0.63,1}
\definecolor{Backquotes}{rgb}{0,0,0}
\definecolor{Classname}{rgb}{0,0,0}
\definecolor{FunctionName}{rgb}{0,0,0}
\definecolor{Operators}{rgb}{0,0,0}
\definecolor{Background}{rgb}{0.98,0.98,0.98}
\lstdefinelanguage{Python}{
numbers=left,
numberstyle=\footnotesize,
numbersep=1em,
xleftmargin=0em,
xrightmargin=-3em,
framextopmargin=2em,
framexbottommargin=2em,
showspaces=false,
showtabs=false,
showstringspaces=false,
frame=l,
tabsize=2,
% Basic
basicstyle=\ttfamily\small\setstretch{1},
%backgroundcolor=\color{Background},
% Comments
commentstyle=\color{Comments}\slshape,
% Strings
stringstyle=\color{Strings},
morecomment=[s][\color{Strings}]{"""}{"""},
morecomment=[s][\color{Strings}]{'''}{'''},
% keywords
morekeywords={import,from,class,def,for,while,if,is,in,elif,else,not,and,or,print,break,continue,return,True,False,None,access,as,,del,except,exec,finally,global,import,lambda,pass,print,raise,try,assert},
keywordstyle={\color{Keywords}\bfseries},
% additional keywords
morekeywords={[2]@invariant,pylab,numpy,np,scipy},
keywordstyle={[2]\color{Decorators}\slshape},
emph={self},
emphstyle={\color{self}\slshape},
texcl=true,
%
}
\usepackage{caption}
\DeclareCaptionFont{white}{\color{white}}
\DeclareCaptionFormat{listing}{\colorbox{gray}{\parbox{\textwidth}{#1#2#3}}}
\captionsetup[lstlisting]{format=listing,labelfont=white,textfont=white} 


\usepackage{cmap} % Улучшенный поиск русских слов в полученном pdf-файле
\usepackage[T2A]{fontenc} % Поддержка русских букв
\usepackage[utf8]{inputenc} % Кодировка utf8
\usepackage[english, russian]{babel} % Языки: русский, английский


 

								% TeX will automatically convert eps --> pdf in pdflatex		
\usepackage{amssymb}



\begin{document}
\begin{titlepage}

\thispagestyle{empty}

\begin{center}
Федеральное государственное бюджетное образовательное учреждение высшего профессионального образования Московский государственный технический университет имени Н.Э. Баумана
\end{center}


\vfill

\centerline{\large{Лабораторная работа №2}}
\centerline{\large{<<Метод наименьших квадратов>>}}
\centerline{\large{по дисциплине}}
\centerline{\large{<<Численные методы>>}}

\vfill

Студент группы ИУ9-62 \hfill Белогуров А.А.

Преподаватель \hfill Домрачева А.Б.
\vfill

\centerline{Москва, 2017}
\clearpage
\end{titlepage}

\newpage
\setcounter{page}{2}
 \textbf{Постановка задачи:}
 
 Функция $y_i = f(x_i),  i = \overline{1,n}$ задана таблично \eqref{tab:1}, исходные данные включают ошибки измерения.

\begin{table}[h]
\caption{\label{tab:1}}
\begin{center}
\begin{tabular}{|c|c|c|c|}
\hline
$x_1$ & $x_2$ & ... & $x_n$ \\
\hline
$y_1$ & $y_2$ & ... & $y_n$ \\
\hline
\end{tabular}
\end{center}
\end{table}

Найти функцию $y_i \approx f(x_i), i = \overline{1,n}$ удовлетворяющую условию \eqref{eq:1}, без учета вычислительной погрешности \eqref{eq:2}:

\begin{equation}\label{eq:1}
\sum_{i = 1}^{n} (y_i - f(x_i))^2 \to min
\end{equation}

\begin{equation}\label{eq:2}
\sum_{i = 1}^{n} (y_i - f(x_i))^2 = 0
\end{equation}
\\\\

\textbf{Теоретические сведения:}

В качестве $f(x_i)$ возьмем следующую функцию \eqref{eq:3}:

\begin{equation}\label{eq:3}
f(a, b, x_i) = f(x_i) = ax_i + b 
\end{equation}
\begin{equation*}
f(a, b, x) = \sum_{i = 1}^n (y_i - ax_i - b)^2 \to min
\end{equation*}

Задача поиска экстремума заменяется эквивалентной задачей равенства нулю производной исходной функции, так как переменные $a$ и $b$ являются неизвестными \eqref{eq:4} и \eqref{eq:5}:

\begin{equation}\label{eq:4}
\begin{cases}
\frac {\partial f(a, b, x)}{\partial a} = 0 
\\
\frac {\partial f(a, b, x)}{\partial b} = 0 
\end{cases}
\Rightarrow 
\begin{cases}
\sum^{n}_{i = 1} (y_i - ax_i - b)x_i = 0 
\\
\sum^{n}_{i = 1} (y_i - ax_i - b) = 0 
\end{cases}
\Rightarrow 
\end{equation}
 
\begin{equation}\label{eq:5}
\Rightarrow
\begin{cases}
a\sum^n_{i=1}x^2_i + b\sum^n_{i=1}x_i  = \sum^n_{i=1}x_i y_i 
\\
a\sum^n_{i=1}x_i + bn  = \sum^n_{i=1}y_i 
\end{cases}
\end{equation}

Подставим замену \eqref{eq:6} в \eqref{eq:5} и получим следующую систему \eqref{eq:7}.

\begin{equation}\label{eq:6}
A = \sum^n_{i=1}x^2_i, \quad
B = \sum^n_{i=1}x_i, \quad
C  = \sum^n_{i=1}x_i y_i, \quad
D = \sum^n_{i=1}y_i
\end{equation}


\begin{equation}\label{eq:7}
\begin{cases}
aA+ bB  = C
\\
aB + bn  = D
\end{cases}
\end{equation}

Выразим $b$ и подставим в систему \eqref{eq:7} для нахождения $a$ \eqref{eq:8}:

\begin{equation}\label{eq:8}
b = \frac {D-aB}{n}, \quad
a = \frac {Cn-DB}{nA-B^2}
\end{equation}

В качестве исходных данных возьмем функцию $x = y$, для каждой координаты $x$  и $y$ введем погрешность $\varepsilon = 0.1$, $n = 10$.
\\\\

\textbf{Практическая реализация:}
\hypertarget{lst:quad}{}
\lstinputlisting[label = lst:quad,  caption=Метод наименьших квадратов, language=Python]{lab2.py}


\textbf{Результаты:}

При каждом запуске программы \hyperlink{lst:quad}{(Листинг 1)} координаты $x$ и $y$ будут разными, так как мы считаем их методом $random$. Ниже приведен один из возможных ответов программы:

\begin{equation*}
a: 0.9971347394546591
 \end{equation*}
\begin{equation*}
b: 0.0150910246109411
\end{equation*}
\\\\

\textbf{Выводы:}

В ходе выполнения лабораторной работы был рассмотрен <<Метод наименьших квадратов>> и разобрана задача, которая является эквивалентной по отношению к исходной задаче. Была написано реализация данного метода на языке программирования Python.

В качестве результатов были получены два рациональных числа $a$ и $b$. Погрешность для этих чисел получилась $\varepsilon = 0.01$, которая обусловлена вычислительной погрешностью и погрешностью метода.


 
 
 
 
 
 
 \enddocument
 
 