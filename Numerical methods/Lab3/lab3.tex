% !TEX encoding = UTF-8 Unicode
\documentclass[a4paper, 12pt]{article}   	% use "amsart" instead of "article" for AMSLaTeX format
\usepackage[left=20mm, top=15mm, right=15mm, bottom=15mm, nohead, footskip=10mm]{geometry}    

            
\usepackage[parfill]{parskip}    		% Activate to begin paragraphs with an empty line rather than an indent
\usepackage{graphicx}				% Use pdf, png, jpg, or eps§ with pdflatex; use eps in DVI mode
\usepackage[14pt]{extsizes}
\usepackage{setspace,amsmath}
\usepackage{ dsfont }
\usepackage{amsmath,amssymb}
\usepackage{hyperref}

\usepackage{array}
\newcolumntype{P}[1]{>{\centering\arraybackslash}p{#1}}

\usepackage{xcolor}
\usepackage{color}
\usepackage{listings}
\usepackage{setspace}
\definecolor{Code}{rgb}{0,0,0}
\definecolor{Decorators}{rgb}{0.5,0.5,0.5}
\definecolor{Numbers}{rgb}{0.5,0,0}
\definecolor{MatchingBrackets}{rgb}{0.25,0.5,0.5}
\definecolor{Keywords}{rgb}{0,0,1}
\definecolor{self}{rgb}{0,0,0}
\definecolor{Strings}{rgb}{0,0.63,0}
\definecolor{Comments}{rgb}{0,0.63,1}
\definecolor{Backquotes}{rgb}{0,0,0}
\definecolor{Classname}{rgb}{0,0,0}
\definecolor{FunctionName}{rgb}{0,0,0}
\definecolor{Operators}{rgb}{0,0,0}
\definecolor{Background}{rgb}{0.98,0.98,0.98}
\lstdefinelanguage{Python}{
numbers=left,
numberstyle=\footnotesize,
numbersep=1em,
xleftmargin=0em,
xrightmargin=-3em,
framextopmargin=2em,
framexbottommargin=2em,
showspaces=false,
showtabs=false,
showstringspaces=false,
frame=l,
tabsize=2,
% Basic
basicstyle=\ttfamily\small\setstretch{1},
%backgroundcolor=\color{Background},
% Comments
commentstyle=\color{Comments}\slshape,
% Strings
stringstyle=\color{Strings},
morecomment=[s][\color{Strings}]{"""}{"""},
morecomment=[s][\color{Strings}]{'''}{'''},
% keywords
morekeywords={import,from,class,def,for,while,if,is,in,elif,else,not,and,or,print,break,continue,return,True,False,None,access,as,,del,except,exec,finally,global,import,lambda,pass,print,raise,try,assert},
keywordstyle={\color{Keywords}\bfseries},
% additional keywords
morekeywords={[2]@invariant,pylab,numpy,np,scipy},
keywordstyle={[2]\color{Decorators}\slshape},
emph={self},
emphstyle={\color{self}\slshape},
texcl=true,
%
}
\usepackage{caption}
\DeclareCaptionFont{white}{\color{white}}
\DeclareCaptionFormat{listing}{\colorbox{gray}{\parbox{\textwidth}{#1#2#3}}}
\captionsetup[lstlisting]{format=listing,labelfont=white,textfont=white} 


\usepackage{cmap} % Улучшенный поиск русских слов в полученном pdf-файле
\usepackage[T2A]{fontenc} % Поддержка русских букв
\usepackage[utf8]{inputenc} % Кодировка utf8
\usepackage[english, russian]{babel} % Языки: русский, английский


 

								% TeX will automatically convert eps --> pdf in pdflatex		
\usepackage{amssymb}



\begin{document}
\begin{titlepage}

\thispagestyle{empty}

\begin{center}
Федеральное государственное бюджетное образовательное учреждение высшего профессионального образования Московский государственный технический университет имени Н.Э. Баумана
\end{center}


\vfill

\centerline{\large{Лабораторная работа №3}}
\centerline{\large{<<Численное интегрирование>>}}
\centerline{\large{по дисциплине}}
\centerline{\large{<<Численные методы>>}}

\vfill

Студент группы ИУ9-62 \hfill Белогуров А.А.

Преподаватель \hfill Домрачева А.Б.
\vfill

\centerline{Москва, 2017}
\clearpage
\end{titlepage}

\newpage
\setcounter{page}{2}

\textbf{Цель:}
Сравнительный анализ методов численного интегрирования: 
\begin{enumerate}
\item Метод средних прямоугольников.
\item Метод квадраторных трапеций.
\item Метод Симпсона.
\\\\
\end{enumerate}


 
 \textbf{Постановка задачи:}
 
 Дан интеграл:
 $$\int_{a}^{b} f(x)dx$$где $a, b, f(x)$ - известны.
 
 Найти интеграл:
 $$I \approx I^* $$
 
 Тестовый пример \eqref{eq:testing}:
 \begin{equation}\label{eq:testing}
 I = \int_{0}^{1} e^x dx = e^x \Big|_{0}^{1} = e - 1 = 1.718282
 \end{equation}
\\\\

\textbf{Теоретические сведения:}

Численное интегрирование — вычисление значения определённого интеграла (как правило, приближённое). Под численным интегрированием понимают набор численных методов для нахождения значения определённого интеграла.

Основная идея большинства методов численного интегрирования состоит в замене подынтегральной функции на более простую, интеграл от которой легко вычисляется аналитически. При этом для оценки значения интеграла получаются формулы вида:
$$I \approx \sum_{i=1}^{n} w_i f(x_i)$$

При замене подынтегральной функции на полином нулевой, первой и второй степени получаются соответственно методы прямоугольников, трапеций и парабол (Симпсона).

\textbf{1. Метод средних прямоугольников}

Если рассмотреть график подынтегральной функции, то метод будет заключаться в приближённом вычислении площади под графиком суммированием площадей конечного числа прямоугольников, ширина которых будет определяться расстоянием между соответствующими соседними узлами интегрирования, а высота — значением подынтегральной функции в этих узлах.

Положим, что $h = \frac{a-b}{n}$ тогда получим следующее разбиение отрезка $[a, b]$ на $n$ равных частей \eqref{eq:razbienie}:

\begin{equation}\label{eq:razbienie}
x_0 = a, x_1 = x_0 + h, x_2 = x_1 + h, ..., x_n = x_{n-1} + h
\end{equation}


Тогда приближенное значение интеграла с помощью метода средних прямоугольников будет вычисляться по формуле \eqref{eq:rectangle}:

\begin{equation}\label{eq:rectangle}
I^* = \sum_{i=1}^{n} I_{i}^{*} =h \sum_{i=1}^{n} f (\frac{x_{i-1} + x_i}{2}) = h \sum_{i=1}^{n} f(a + i * h - \frac{h}{2})
\end{equation}

\textbf{2. Метод трапеций}

Метод численного интегрирования функции одной переменной, заключающийся в замене на каждом элементарном отрезке подынтегральной функции на многочлен первой степени, то есть линейную функцию. 

Разбиение производится аналогично методу средних прямоугольников \eqref{eq:razbienie}.

Квадраторная формула трапеций будет выглядеть следующим образом \eqref{eq:trapezoid1}, \eqref{eq:trapezoid2}:
\begin{equation}\label{eq:trapezoid1}
I^*_i = \frac{f(x_{i-1})+f(x_i)}{2} h
\end{equation}
\begin{equation}\label{eq:trapezoid2}
I^* = \sum_{i=1}^{n} I_{i}^{*} = \frac{h}{2} \sum_{i=1}^{n} (f(x_0) + f(x_1) + ... + f(x_n)) = h (\frac{f(a)+f(b)}{2} + \sum_{i=1}^{n-1} f(x_i))
\end{equation}

\textbf{3. Метод Симпсона}

Суть метода заключается в приближении подынтегральной функции на отрезке $[a, b]$
интерполяционным многочленом второй степени $P_2(x)$, то есть приближение графика функции на отрезке параболой.

Разбиение производится аналогично методу средних прямоугольников \eqref{eq:razbienie}.

Парабола Симпсона представлена формулой \eqref{eq:parsimpson}:
\begin{multline}\label{eq:parsimpson}
P_2(x) = f(x_{i-0.5}) + \frac{f(x_i) - f(x_{i-1})}{h}(x-x_{i-0.5}) + \\ 
+ \frac{f(x_i) - 2f(x_{i-0.5}) + f(x_{i-1})}{\frac{h^2}{2}}*(x-x_{i-0.5})^2
\end{multline}

Тогда формула Симпсона выглядит следующим образом\eqref{eq:simpson1}, \eqref{eq:simpson2}:
\begin{equation}\label{eq:simpson1}
I^*_i = \int_{x_{i-1}}^{x_i} P_2(x)dx = \frac{h}{6} (f(x_{i-1}) + 4 f(x_{i-0.5}) + f(x_i))
\end{equation}
\begin{equation}\label{eq:simpson2}
I* = \sum_{i=1}^{n} I^*_i = \frac{h}{6} (f(a) + f(b) + 4\sum_{i=1}^n f(x_{i-0.5}) + 2\sum_{i=1}^{n-1}f(x_i))
\end{equation}
\\\\

\textbf{Вычисление интегралов с учетом погрешности}

$I = I^* + O(h^k)$, где $k$ - порядок точности метода.

$k = 2$ - для метода средних прямоугольников и трапеций.

$k = 4$ - для метода Симпсона.

$O(h^k) = ch^k$, где $h$ - шаг, $c$ - некоторая константа.

Тогда:

\begin{equation*}
I \approx I^*_h + ch^k
\end{equation*}

Считаем, что вычисления проводятся без вычислительной погрешности, тогда можно записать строгое равенство:
\begin{equation}\label{eq:pogr1}
I = I^*_h + ch^k
\end{equation}
\begin{equation}\label{eq:pogr2}
I = I^*_\frac{h}{2} + c{(\frac{h}{2})}^k
\end{equation}

Из равенств \eqref{eq:pogr1} и \eqref{eq:pogr2} следует равенство \eqref{eq:pogr3}:
\begin{equation}\label{eq:pogr3}
I^*_h + ch^k = I^*_\frac{h}{2} + c{(\frac{h}{2})}^k
\end{equation}
\begin{equation}\label{eq:pogr4}
c{(\frac{h}{2})}^k = \frac{I^*_h - I^*_\frac{h}{2}}{1-2^k}
\end{equation}

Подставим \eqref{eq:pogr4} в \eqref{eq:pogr1} и получим значение интеграла с погрешностью\eqref{eq:pogr5}:
\begin{equation}\label{eq:pogr5}
I = I^*_\frac{h}{2} + \frac{I^*_h - I^*_\frac{h}{2}}{1-2^k}
\end{equation}

Где значение $R$ \eqref{eq:pogr6} есть уточнение по Ричардсону:
 \begin{equation}\label{eq:pogr6}
R = \frac{I^*_h - I^*_\frac{h}{2}}{1-2^k}
\end{equation}

\textbf{Практическая реализация:}
\hypertarget{lst:quad}{}
\lstinputlisting[label = lst:quad,  caption=Численное интегрирование, language=Python]{lab3.py}


\textbf{Результаты:}

Для тестирования был выбран интеграл \eqref{eq:testing}.

В качестве $\varepsilon$ для каждого метода были выбраны следующие значения:
$$ \varepsilon = 0.1, \varepsilon = 0.01, \varepsilon = 0.001 $$

Ниже приведена таблица результата тестовой программы \hyperlink{lst:quad}{(Листинг 1)} на указанных выше методах:
\begin{table}[h]
\begin{center}
\begin{tabular}{|P{4cm}|P{2.5cm}|P{4cm}|P{4cm}|}
\hline
Метод & Кол-во итераций & Значение без уточнения по Ричардсону & Значение с уточнением по Ричардсону \\
\hline
\multicolumn{4}{|c|}{$\varepsilon = 0.1$} \\
\hline
Метод средних прямоугольников & 2 & 1.713815 & 1.718249 \\
\hline
Метод трапеций & 2 & 1.727222 & 1.718319 \\
\hline
Метод Симпсона & 2 & 1.718284 & 1.718282 \\
\hline
\multicolumn{4}{|c|}{$\varepsilon = 0.01$} \\
\hline
Метод средних прямоугольников & 2 & 1.713815 & 1.718249 \\
\hline
Метод трапеций & 2 & 1.727222 & 1.718319 \\
\hline
Метод Симпсона & 2 & 1.718284 & 1.718282 \\
\hline
\multicolumn{4}{|c|}{$\varepsilon = 0.001$} \\
\hline
Метод средних прямоугольников & 4 & 1.718002 & 1.718282 \\
\hline
Метод трапеций & 4 & 1.718841 & 1.718282 \\
\hline
Метод Симпсона & 2 & 1.718284 & 1.718282 \\
\hline
\end{tabular}
\end{center}
\end{table}
\\\\


\textbf{Выводы:}

В ходе выполнения лабораторной работы были рассмотрены различных 3 метода численного интегрирования: метод средних прямоугольников, метод трапеций и метод Симпсона. Была написана реализация данных методов на языке программирования Python.

Как видно выше в таблице вычислений, среди всех методов выигрывает метод Симпсона (он точнее и для вычислений при $\varepsilon = 0.001$ ему понадобилось меньше итераций для получения результата).

Если же сравнивать между собой два остальных метода, то метод средних прямоугольников является точнее метода трапеций. Но в то же время метод трапеций может применяться с произвольным шагом, в отличие от метода средних прямоугольников, который не применим, например, к функциям, заданным в конечном числе точек.




 
 
 
 
 
 
 \enddocument