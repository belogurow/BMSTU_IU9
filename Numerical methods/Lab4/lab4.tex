% !TEX encoding = UTF-8 Unicode
\documentclass[a4paper, 12pt]{article}   	% use "amsart" instead of "article" for AMSLaTeX format
\usepackage[left=20mm, top=15mm, right=15mm, bottom=15mm, nohead, footskip=10mm]{geometry}    

            
\usepackage[parfill]{parskip}    		% Activate to begin paragraphs with an empty line rather than an indent
\usepackage{graphicx}				% Use pdf, png, jpg, or eps§ with pdflatex; use eps in DVI mode
\usepackage[14pt]{extsizes}
\usepackage{setspace,amsmath}
\usepackage{ dsfont }
\usepackage{amsmath,amssymb}
\usepackage{hyperref}
\usepackage{framed}

\usepackage{array}
\newcolumntype{P}[1]{>{\centering\arraybackslash}p{#1}}

\usepackage{xcolor}
\usepackage{color}
\usepackage{listings}
\usepackage{setspace}
\definecolor{Code}{rgb}{0,0,0}
\definecolor{Decorators}{rgb}{0.5,0.5,0.5}
\definecolor{Numbers}{rgb}{0.5,0,0}
\definecolor{MatchingBrackets}{rgb}{0.25,0.5,0.5}
\definecolor{Keywords}{rgb}{0,0,1}
\definecolor{self}{rgb}{0,0,0}
\definecolor{Strings}{rgb}{0,0.63,0}
\definecolor{Comments}{rgb}{0,0.63,1}
\definecolor{Backquotes}{rgb}{0,0,0}
\definecolor{Classname}{rgb}{0,0,0}
\definecolor{FunctionName}{rgb}{0,0,0}
\definecolor{Operators}{rgb}{0,0,0}
\definecolor{Background}{rgb}{0.98,0.98,0.98}
\lstdefinelanguage{Python}{
numbers=left,
numberstyle=\footnotesize,
numbersep=1em,
xleftmargin=0em,
xrightmargin=-3em,
framextopmargin=2em,
framexbottommargin=2em,
showspaces=false,
showtabs=false,
showstringspaces=false,
frame=l,
tabsize=2,
% Basic
basicstyle=\ttfamily\small\setstretch{1},
%backgroundcolor=\color{Background},
% Comments
commentstyle=\color{Comments}\slshape,
% Strings
stringstyle=\color{Strings},
morecomment=[s][\color{Strings}]{"""}{"""},
morecomment=[s][\color{Strings}]{'''}{'''},
% keywords
morekeywords={import,from,class,def,for,while,if,is,in,elif,else,not,and,or,print,break,continue,return,True,False,None,access,as,,del,except,exec,finally,global,import,lambda,pass,print,raise,try,assert},
keywordstyle={\color{Keywords}\bfseries},
% additional keywords
morekeywords={[2]@invariant,pylab,numpy,np,scipy},
keywordstyle={[2]\color{Decorators}\slshape},
emph={self},
emphstyle={\color{self}\slshape},
texcl=true,
%
}
\usepackage{caption}
\DeclareCaptionFont{white}{\color{white}}
\DeclareCaptionFormat{listing}{\colorbox{gray}{\parbox{\textwidth}{#1#2#3}}}
\captionsetup[lstlisting]{format=listing,labelfont=white,textfont=white} 


\usepackage{cmap} % Улучшенный поиск русских слов в полученном pdf-файле
\usepackage[T2A]{fontenc} % Поддержка русских букв
\usepackage[utf8]{inputenc} % Кодировка utf8
\usepackage[english, russian]{babel} % Языки: русский, английский


 

								% TeX will automatically convert eps --> pdf in pdflatex		
\usepackage{amssymb}



\begin{document}
\begin{titlepage}

\thispagestyle{empty}

\begin{center}
Федеральное государственное бюджетное образовательное учреждение высшего профессионального образования Московский государственный технический университет имени Н.Э. Баумана
\end{center}


\vfill

\centerline{\large{Лабораторная работа №4}}
\centerline{\large{<<Сравнение методов решения уравнения $f(x) =0$>>}}
\centerline{\large{по дисциплине}}
\centerline{\large{<<Численные методы>>}}

\vfill

Студент группы ИУ9-62 \hfill Белогуров А.А.

Преподаватель \hfill Домрачева А.Б.
\vfill

\centerline{Москва, 2017}
\clearpage
\end{titlepage}

\newpage
\setcounter{page}{2}

\textbf{Цель:}
Сравнительный анализ методов решения уравнения $f(x)$: 
\begin{enumerate}
\item Метод деления отрезка пополам.
\item Метод золотого сечения.
\\\\
\end{enumerate}


 
 \textbf{Постановка задачи:}
 
 Дано нелинейное уравнение $f(x) = 0$ третьей степени, которое выбирается произвольно. В моем случае было выбрано следующее ур-е \eqref{eq:testing}:
 \begin{equation}\label{eq:testing}
 f(x) = 5x^3 - 8x^2 - 8x + 5 = 0
 \end{equation}
 
 C помощью выбранных методов получить приближенное решение ур-я \eqref{eq:testing} с заданной точностью $\varepsilon$ и сравнить полученные результаты.
 
 Функция $f(x)$ должна быть непрерывна.
\\\\

\textbf{Теоретические сведения:}

\textbf{1. Метод деления отрезка пополам}

Метод деления отрезка пополам (метод бисекции) - один из методов поиска корней нелинейного уравнения, которое основывается на разбиении отрезка пополам на каждом шаге алгоритма.

\textbf{Описание алгоритма: }
Для начала вычислений необходимо знать значения отрезка \eqref{eq:bis1}, на концах которого функция принимает значения разных знаков (следствие из теоремы Больцано-Коши)
\begin{equation}\label{eq:bis1}
[x_L, x_R]
\end{equation}
\begin{framed}
\textbf{Cледствие из теоремы Больцано-Коши:}

Пусть непрерывная функция $f(x) \in C([a, b])$, тогда если \\$sign(f(a)) \neq sign(f(b))$, то 
$\exists с \in [a, b]: f(c) = 0.$
\end{framed}

Для того, чтобы определить, принимает ли функция разные значения на концах отрезка - можно использовать два способа: через умножение \eqref{eq:bis2} и через обычное сравнение \eqref{eq:bis3}:
\begin{equation}\label{eq:bis2}
 f(x_L) \times f(x_R) < 0
 \end{equation}
 \begin{equation}\label{eq:bis3}
 sign(f(x_L)) \neq sign(f(x_R))
 \end{equation}
 
При использовании формулы \eqref{eq:bis2} может возникнуть преждевременное переполнение, поэтому для устранения этой проблемы и для увеличения быстродействия будет использоваться формула \eqref{eq:bis3}.

Из непрерывности функции $f(x)$ и условия \eqref{eq:bis3} следует, что на отрезке \eqref{eq:bis1} существует хотя бы один корень уравнения. В нашем случае уравнение третьей степени, поэтому мы должны выделить <<вилки>> - отрезки, включающие точку, в котором график пересекает ось $oX$ (решение). Таким образом надо будет выделить три таких отрезка.

Далее вычисляется значение $x_M$ \eqref{eq:bis4} в середине отрезка:
\begin{equation}\label{eq:bis4}
 \frac{x_L+x_R}{2}
 \end{equation}
 
 \begin{itemize}
\item Если $|b - a| \leq 2 \times \varepsilon$, то корень $x = \frac{b-a}{2}$ найден
\item Иначе разобьем отрезок \eqref{eq:bis1} на два равных отрезка \eqref{eq:bis5}, \eqref{eq:bis6}:
\end{itemize}
\begin{equation}\label{eq:bis5}
[x_L, x_M]
\end{equation}
\begin{equation}\label{eq:bis6}
[x_M, x_R]
\end{equation}

Определим отрезок, на котором функция меняет свой знак:
\begin{itemize}
 \item Если условие \eqref{eq:bis3} выполняется на левом отрезке \eqref{eq:bis5}, то, соответственно, корень находится на левом отрезке. Производим замену $x_R = x_M$ и возвращаемся к шагу разбиения отрезка пополам \eqref{eq:bis4}
 \item Если условие \eqref{eq:bis3} выполняется на правом отрезке \eqref{eq:bis6}, то, соответственно, корень находится на правом отрезке. Производим замену $x_L = x_M$ и возвращаемся к шагу разбиения отрезка пополам \eqref{eq:bis4}
 \end{itemize}

\textbf{2. Метод золотого сечения}

В основе метода лежит принцип деления отрезка пополам в пропорциях золотого сечения. 

Аналогично предыдущему методу выбирается отрезок \eqref{eq:golden1}, внутри которого функция $f(x)$ принимает нулевые значения:
\begin{equation}\label{eq:golden1}
[a, b]
\end{equation}

Необходимо выбрать две точки $x_1$ и $x_2$ такие, что:
\begin{equation}\label{eq:golden2}
\frac{b - a}{b-x_1} = \frac{b-a}{x_2-a}=\varphi = \frac{1 + \sqrt{5}}{2}
\end{equation}

Где $\varphi$ \eqref{eq:golden2} - пропорция золотого сечения. Таким образом точки $x_1$ \eqref{eq:golden3} и $x_2$ \eqref{eq:golden4} определяются следующим образом:
\begin{equation}\label{eq:golden3}
x_1 = b - \frac{(b-a)}{\varphi}
\end{equation}
\begin{equation}\label{eq:golden4}
x_2 = a + \frac{(b-a)}{\varphi}
\end{equation}

\textbf{Описание алгоритма: }
Алгоритм почти идентичен алгоритму метода деления отрезка пополам. Отличие заключается в том, что отрезок делится не пополам, а на три части в пропорциях золотого сечения.

Таким образом, в начале проверяется:
 \begin{itemize}
 \item Если $|b - a| \leq 2 \times \varepsilon$, то корень $x = \frac{b-a}{2}$ найден
 \item Иначе находятся значения точек $x_1$ \eqref{eq:golden3} и $x_2$ \eqref{eq:golden4}
 \end{itemize}
 
 Определим отрезок, на котором функция меняет свой знак:
\begin{itemize}
 \item Если условие \eqref{eq:bis3} выполняется на левом отрезке $[a, x_1]$, то, соответственно, корень находится на левом отрезке. Производим замену $b = x_1$ и возвращаемся к предыдущему шагу
 \item Если условие \eqref{eq:bis3} выполняется на отрезке посередине $[x_1, x_2]$, то, соответственно, корень находится на отрезке посередине. Производим замену $a = x_1$, $b = x_2$ и возвращаемся к предыдущему шагу 
 \item Если условие \eqref{eq:bis3} выполняется на правом отрезке $[x_2, b]$, то, соответственно, корень находится на правом отрезке. Производим замену $a = x_2$ и возвращаемся к предыдущему шагу
 \end{itemize}

\textbf{Практическая реализация:}
\hypertarget{lst:quad}{}
\lstinputlisting[label = lst:quad,  caption=Методы решения уравнения f(x) = 0, language=Python]{lab4.py}


\textbf{Результаты:}

Для тестирования было выбрано уравнение \eqref{eq:testing}, $\varepsilon = 0.001$.

Уравнение \eqref{eq:testing} имеет три корня, для корректного результата выберем три одинаковых <<ветки>>:
\begin{table}[h]
\begin{center}
\begin{tabular}{|P{4cm}|P{2.5cm}|P{4cm}|}
\hline
[-1.5, 0] & [0, 1.5] & [1.5, 3] \\
\hline
\end{tabular}
\end{center}
\end{table}

Ниже приведена таблица результата тестовой программы \hyperlink{lst:quad}{(Листинг 1)} на указанных выше методах:
\begin{table}[h]
\begin{center}
\begin{tabular}{|P{4cm}|P{4cm}|P{4cm}|}
\hline
Отрезок & Значение & Количество итераций \\
\hline
\multicolumn{3}{|c|} {Метод деления отрезка пополам}\\
\hline
[-1.5, 0] & -0.999756 & 10  \\
\hline
 [0, 1.5] & 0.469482  & 10 \\
\hline
[1.5, 3]  & 2.130615 & 10 \\
\hline
\multicolumn{3}{|c|} {Метод золотого сечения}\\
\hline
[-1.5, 0] & -0.999337 & 7  \\
\hline
 [0, 1.5] & 0.468734  & 6 \\
\hline
[1.5, 3]  & 2.130709 & 6 \\
\hline
\multicolumn{3}{|c|} {Значения, полученные с помощью сервиса WolframAlpha}\\
\hline
[-1.5, 0] & -1 & -  \\
\hline
 [0, 1.5] & 0.469338  & - \\
\hline
[1.5, 3]  & 2.130662 & - \\
\hline
\end{tabular}
\end{center}
\end{table}
\\\\


\textbf{Выводы:}

В ходе выполнения лабораторной работы были рассмотрены 2 различных метода нахождения корня нелинейного уравнения: метод деления отрезка пополам и метод золотого сечения. Была написана реализация данных методов на языке программирования Python.

Как видно выше в таблице вычислений, по количеству итераций выигрывает метод золотого сечения, так как он в своей реализации делит отрезок на 3 части и, следовательно, приходит быстрее к результату, однако если сравнивать оба метода по точности вычислений, то ближе к правильным ответам оказался именно метод деления отрезка пополам. Стоит заметить, что при округлении значений с заданной точностью $\varepsilon$ оба методы будут верны, поэтому в данном случае не будет корректным сравнивать эти методы по точности вычислений. Одна из причин, почему метод деления оказался ближе к результату, чем метод золотого сечения, заключается в различном способе нахождения конечного результата. Однако, если целью сравнения будет именно поиск более точного метода, то стоит рассматривать не одно уравнение, а большое их количество.

 
 \enddocument