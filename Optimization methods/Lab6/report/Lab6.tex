%% It is just an empty TeX file.
%% Write your code here.
% !TEX encoding = UTF-8 Unicode
\documentclass[a4paper, 12pt]{article}   	% use "amsart" instead of "article" for AMSLaTeX format
\usepackage[left=20mm, top=15mm, right=10mm, bottom=15mm]{geometry}    

            
\usepackage[parfill]{parskip}    		% Activate to begin paragraphs with an empty line rather than an indent
\usepackage{graphicx}				% Use pdf, png, jpg, or eps§ with pdflatex; use eps in DVI mode
\usepackage[14pt]{extsizes}
\usepackage{setspace,amsmath}
\usepackage{mathtools}
\usepackage{ dsfont }
\usepackage{amsmath,amssymb}
\usepackage[unicode]{hyperref}

\usepackage{xcolor}
\usepackage{color}
\usepackage{minted}
\usepackage{caption}

\usepackage{array}
\newcolumntype{P}[1]{>{\centering\arraybackslash}p{#1}}

\usepackage{cmap} % Улучшенный поиск русских слов в полученном pdf-файле
\usepackage[T2A]{fontenc} % Поддержка русских букв
\usepackage[utf8]{inputenc} % Кодировка utf8
\usepackage[english, russian]{babel} % Языки: русский, английский

								% TeX will automatically convert eps --> pdf in pdflatex		
\usepackage{amssymb}

\begin{document}
\begin{titlepage}

\thispagestyle{empty}

\begin{center}
Федеральное государственное бюджетное образовательное учреждение высшего профессионального образования Московский государственный технический университет имени Н.Э. Баумана
\end{center}


\vfill

\centerline{\large{Лабораторная работа №6. Вариант 1.}}

\centerline{\large{«Методы решения задач линейного и}} 
\centerline{\large{линейного целочисленного программирования»}} 

\centerline{\large{по курсу}}
\centerline{\large{«Методы оптимизации»}}


\vfill

Студент группы ИУ9-82 \hfill Белогуров А.А.

Преподаватель \hfill Каганов Ю.T. 
\vfill

\centerline{Москва, 2018}
\clearpage
\end{titlepage}

\newpage
\setcounter{page}{2}

\tableofcontents

\newpage

\section{Цель работы}

\begin{enumerate}
    \item Изучение симплекс-метода линейного программирования и методов линейного целочисленного программирования.
    \item Разработка программы реализации алгоритма симплекс-метода линейного программирования и алгоритмов линейного целочисленного программирования.
    \item Решение задачи линейного и линейного целочисленного программирования. 
\end{enumerate}

\newpage

\section{Постановка задачи}
    \textbf {Дано}: 1 Вариант. Функция:
    \begin{equation}
        f(x) = x_1 - x_2 + x_3.
    \end{equation}
    Функции ограничений:
    \begin{equation}
        \begin{cases}
            -x_1 + 2x_2 - x_3 = 4 \\
            3 x_1 + x_2 + x_4 = 14 \\
            x_1, ..., x_4 \geq 0
        \end{cases}
    \end{equation}
    
\subsection{Задача 6.1}

    \begin{enumerate}
        \item Найти условный максимум для задачи линейного программирования симплекс-методом Дж. Данцига.
        \item Реализовать алгоритмы с помощью языка программирования высокого уровня.
    \end{enumerate}
    
\subsection{Задача 6.2}

    \begin{enumerate}
        \item Найти условный максимум для задачи линейного целочисленного программирования методом «ветвей и границ».
        \item Реализовать алгоритмы с помощью языка программирования высокого уровня.
    \end{enumerate}
    
    
\newpage
\section{Исследование}
    Найдем условный максимум для функции:
    \begin{equation}
          f(x) = x_1 - x_2 + x_3
    \end{equation}
    с помощью сервиса WolframAlpha.com:
    
     \begin{equation}
        max(f(x)) = 10,\quad (x_1, x_2, x_3, x_4) = (0, 14, 24, 0)
    \end{equation}
    
\subsection{Задача 6.1}

    \subsubsection{Симлекс-метод}
        Это алгоритм решения оптимизационной задачи линейного программирования путём перебора вершин выпуклого многогранника в многомерном пространстве.

        Сущность метода: построение базисных решений, на которых монотонно убывает линейный функционал, до ситуации, когда выполняются необходимые условия локальной оптимальности.

\subsection{Задача 6.2}

    \subsubsection{Метод ветвей и границ}
    Общий алгоритмический метод для нахождения оптимальных решений различных задач оптимизации, особенно дискретной и комбинаторной оптимизации. По существу, метод является вариацией полного перебора с отсевом подмножеств допустимых решений, заведомо не содержащих оптимальных решений.
    
    Общая идея метода может быть описана на примере поиска минимума функции $f(x)$ на множестве допустимых значений переменной $x$. Функция $f$ и переменная $x$ могут быть произвольной природы. Для метода ветвей и границ необходимы две процедуры: ветвление и нахождение оценок (границ).
    
    Так в данном случае необходимо решить задачу целочисленного линейного программирования, то необходимо изменить исходную функцию, так как её решением при первой итерации являются целые числа.
    
    Новая функция для \textbf{Задачи 6.2} будет выглядеть следующим образом:
    \begin{equation}
        f(x) = - x_1 + 11 x_2 - 1 x_3 + 5x_4
    \end{equation}
    
    Функции ограничений не изменяются.
    
\newpage

\section{Практическая реализация}

    Все методы были реализованы на языке программирования \textbf{Python}. 

    \textbf{Листинг 1.} Симлекс-метод.
    \begin{minted}[frame=single, framesep=10pt, fontsize = \footnotesize, linenos=true, breaklines]{python}
class MatrixSimplex:
    def __init__(self, A, b, c, B_idx):
		self.A = np.matrix(A)
		self.b = np.matrix(b)
		self.c = np.matrix(c)

		self.B_idx = B_idx
		self.N_idx = [i for i in range(self.A.shape[1]) if i not in self.B_idx]

	def get_B(self):
		return self.A[:, self.B_idx]

	def get_Cb(self):
		return self.c[:, self.B_idx]

	def get_Cn(self):
		return self.c[:, self.N_idx]

	def get_N(self):
		return self.A[:, self.N_idx]

	def computeZ(self, InvB, N):
		# z = Cb*InvB*b - (Cb*InvB*N - Cn)*Xn
		Cb = self.get_Cb()
		Cb_InvB = np.dot(Cb, InvB)

		Cb_InvB_b = np.dot(Cb_InvB, self.b.T)

		Cb_InvB_N = np.dot(Cb_InvB, N)
		bracket_term = np.subtract(Cb_InvB_N, self.get_Cn())
		return Cb_InvB_b, -1 * bracket_term

	def computeXb(self, InvB, N):
		return np.dot(InvB, self.b.T)

	def is_optimal(self, z):
		for x in np.nditer(z):
			if x > 0:
				return False
		return True

	def compute_entering_variable(self, z):
		max_ascent = 0
		index = -1
		it = np.nditer(z.T, flags=['f_index'])
		while not it.finished:
			if it[0] > max_ascent:
				max_ascent = it[0]
				index = it.index
			it.iternext()
		if index is not -1:  # else returns None
			return self.N_idx[index]

	def replace(self, ev_idx, dv_idx):
		for i in range(len(self.B_idx)):
			if self.B_idx[i] == dv_idx:
				self.B_idx[i] = ev_idx
				break

		self.N_idx = [i for i in range(self.A.shape[1]) if i not in self.B_idx]

	def calculate_value(self, z, Xb):
		print("z = ", z[0, 0])
		for i in range(self.c.shape[1]):
			index = 0
			in_basis = False
			for j in self.B_idx:
				if i == j:
					print("x[{}] = {}".format(j, Xb[index, 0]))
					in_basis = True
				index += 1
			if not in_basis:
				print("x[{}] = {}".format(i, 0))
		print("")
		return

	def do_simplex(self):
		print("\n######## Start Simplex ###########\n")
		while True:
			InvB = np.linalg.inv(self.get_B())
			N = self.get_N()

			z, z_var = self.computeZ(InvB, N)

			Xb = self.computeXb(InvB, N)

			if self.is_optimal(z_var):
				print("Optimal solution found..")
				self.calculate_value(z, Xb)
				return z

			ev_idx = self.compute_entering_variable(z_var)
			if ev_idx is None:
				print("Invalid")
				return 0

			ev = self.A[:, ev_idx]
			dv_idx = self.min_ratio_test(Xb, ev)

			if dv_idx is None:
				print("Invalid")
				return 0

			self.replace(ev_idx, dv_idx)
    \end{minted}

    \textbf{Листинг 2.} Метод ветвей и границ.
    \begin{minted}[frame=single, framesep=10pt, fontsize = \footnotesize, linenos=true, breaklines]{python}
class Node:
	def __init__(self, x_bounds=[], freeze_var_list=[], index=0, upper_or_lower=0):
		self._x_bounds = x_bounds
		self._freeze_var_list = freeze_var_list
		self._index = index
		self._upper_or_lower = upper_or_lower

	def freeze_lower_var(self, index, val):
		self._x_bounds[index] = (None, val)
		self._freeze_var_list.append(index)

	def freeze_upper_var(self, index, val):
		self._x_bounds[index] = (val, None)
		self._freeze_var_list.append(index)

	def freeze_var(self, index, val):
		self._x_bounds[index] = (val, val)
		self._freeze_var_list.append(index)

	def set_lp_res(self, res):
		self._res = res

	def check_integer_var_all_solved(self, m):
		return True if m == len(self._freeze_var_list) else False


def check_all_integers(list):
	is_integer = True
	for i in range(len(list) - 1):
		if not is_int(list[i]):
			is_integer = False
			break

	return is_integer


def calculate(A, b, c):
	global node_counter

	x_bounds = [(0, None) for i in range(len(c))]

	print("\n######## Start B & B ###########\n")

	node = Node(copy.deepcopy(x_bounds), [], node_counter)

	node_counter += 1
	res = solve_LP(x_bounds, A, b, c)

	if check_all_integers(res.x):
		print_result(res)
		return res


	lower = floor(res['x'][integer_var[0]])
	upper = lower + 1

	lower_node = Node(copy.deepcopy(x_bounds), [], node_counter, 1)
	lower_node.freeze_lower_var(integer_var[0], lower)
	add_dangling_node(lower_node, A, b, c)

	node_counter += 1

	upper_node = Node(copy.deepcopy(x_bounds), [], node_counter, 2)
	upper_node.freeze_upper_var(integer_var[0], upper)
	add_dangling_node(upper_node, A, b, c)

	node_counter += 1

	while len(dangling_nodes) > 0:

		index = np.argmin(dangling_nodes_obj)

		x_b = dangling_nodes[index]._x_bounds
		frez = dangling_nodes[index]._freeze_var_list
		res = dangling_nodes[index]._res
		frez_var_index = len(frez)

		u_or_l = dangling_nodes[index]._upper_or_lower
		arbitrary_node = Node(copy.deepcopy(x_b), copy.deepcopy(frez), node_counter, copy.deepcopy(u_or_l))
		u_or_l_b = lower - 1 if (u_or_l == 1) else upper + 1
		arbitrary_node.freeze_var(integer_var[frez_var_index - 1], u_or_l_b)
		x_b_arbi = arbitrary_node._x_bounds
		if check_bounds(x_b_arbi, integer_var[frez_var_index - 1], u_or_l, x_bounds):
			add_dangling_node(arbitrary_node, A, b, c)
		else:
			print("arbitrary Node infeasibile: ", arbitrary_node._index)

		node_counter += 1

		lower = floor(res['x'][integer_var[frez_var_index]])
		upper = lower + 1

		lower_node = Node(copy.deepcopy(x_b), copy.deepcopy(frez), node_counter, 1)
		lower_node.freeze_lower_var(integer_var[frez_var_index], lower)
		add_dangling_node(lower_node, A, b, c)

		node_counter += 1

		upper_node = Node(copy.deepcopy(x_b), copy.deepcopy(frez), node_counter, 2)
		upper_node.freeze_upper_var(integer_var[frez_var_index], upper)
		add_dangling_node(upper_node, A, b, c)

		node_counter += 1

		# убрать break в записке
		# if check_all_integers():
		break

	result = get_max_integer_node()
	print_result(result._res)


def is_int(n):
	return int(n) == float(n)


def solve_LP(x_bounds, A_eq, b_eq, c):
	return Simplex(c, A_eq=A_eq, b_eq=b_eq, bounds=x_bounds)


def print_result(result):
	x_list = result.x
	print("z = {}".format(result.fun))

	for i, item in enumerate(x_list):
		print("x[{}] = {}".format(i, item))

def get_max_integer_node():
	global dangling_nodes
	global dangling_nodes_obj
	global result_Node

	integer_nodes = []
	for node in dangling_nodes:
		if check_all_integers(node._res.x):
			integer_nodes.append(node)

	max_node_result = float("-INF")
	max_node = None
	for node in integer_nodes:
		if node._res.fun >= max_node_result:
			max_node = node
			max_node_result = node._res.fun

	return max_node

def add_dangling_node(node, A, b, c):
	global z_star
	global dangling_nodes
	global dangling_nodes_obj
	global result_Node

	res = solve_LP(node._x_bounds, A, b, c)
	if check_feasibility(res) and res['fun'] > z_star:
		node.set_lp_res(res)

		if check_all_integers(res.x):
			dangling_nodes_obj.append(res['fun'])
			dangling_nodes.append(node)
		elif len(dangling_nodes) == 0:
			dangling_nodes_obj.append(res['fun'])
			dangling_nodes.append(node)
		else:
			return

		if node.check_integer_var_all_solved(len(integer_var)):
			z_star = res['fun']
			result_Node = node
		return True
	else:
		return False
\end{minted}

\newpage

\section{Результаты.}
    Программы, представленные в \textbf{Листинге 1} и \textbf{Листинге 2} дали следующие результаты:
    
    \textbf{Листинг 5.} Результаты выполнения программ.
    \begin{minted}[frame=single, framesep=10pt, fontsize = \footnotesize, linenos=true, breaklines]{text}
Start Simplex method:
    z =  10.0
    x[0] = 0
    x[1] = 14.0
    x[2] = 24.0
    x[3] = 0

Start B&B method
    [(None, 3), (None, 3), (0, None), (0, None)]
    z = 56.0
    x[0] = 2.0
    x[1] = 3.0
    x[2] = 0.0
    x[3] = 5.0
\end{minted}

    Проверка полученных значений для ограничений:
    \begin{equation}
        \begin{cases}
            -x_1 + 2x_2 - x_3 = 4 \\
            3 x_1 + x_2 + x_4 = 14 \\
            x_1, ..., x_4 \geq 0
        \end{cases}
    \end{equation}
    
     \textbf{Симплекс-метод}:
     \begin{equation}
        \begin{cases}
            -1 \times 0 + 2 \times 14 -1 \times 24 + 0 \times 0 = 4 \\
            3 \times 0 + 1 \times 14 + 0 \times 0 + 1 \times 0 = 14 \\
            x_1, ..., x_4 \geq 0
        \end{cases}
    \end{equation}
    Проверка прошла успешно.
    
    \textbf{Метод ветвей и границ}:
    \begin{equation}
        \begin{cases}
            -1 \times 2 + 2 \times 3 -1 \times 0 + 0 \times 5 = 4 \\
            3 \times 2 + 1 \times 3 + 0 \times 0 + 1 \times 5 = 14 \\
            -\infty \leq x_1 \leq 3, \quad -\infty \leq x_2 \leq 3, \quad 0 \leq x_3 \leq \infty, \quad 0 \leq x_4 \leq \infty
        \end{cases}
    \end{equation}
    Проверка прошла успешно.







\end{document} 